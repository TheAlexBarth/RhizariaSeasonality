% Options for packages loaded elsewhere
\PassOptionsToPackage{unicode}{hyperref}
\PassOptionsToPackage{hyphens}{url}
\PassOptionsToPackage{dvipsnames,svgnames,x11names}{xcolor}
%
\documentclass[
  letterpaper,
  DIV=11,
  numbers=noendperiod]{scrartcl}

\usepackage{amsmath,amssymb}
\usepackage{lmodern}
\usepackage{setspace}
\usepackage{iftex}
\ifPDFTeX
  \usepackage[T1]{fontenc}
  \usepackage[utf8]{inputenc}
  \usepackage{textcomp} % provide euro and other symbols
\else % if luatex or xetex
  \usepackage{unicode-math}
  \defaultfontfeatures{Scale=MatchLowercase}
  \defaultfontfeatures[\rmfamily]{Ligatures=TeX,Scale=1}
\fi
% Use upquote if available, for straight quotes in verbatim environments
\IfFileExists{upquote.sty}{\usepackage{upquote}}{}
\IfFileExists{microtype.sty}{% use microtype if available
  \usepackage[]{microtype}
  \UseMicrotypeSet[protrusion]{basicmath} % disable protrusion for tt fonts
}{}
\makeatletter
\@ifundefined{KOMAClassName}{% if non-KOMA class
  \IfFileExists{parskip.sty}{%
    \usepackage{parskip}
  }{% else
    \setlength{\parindent}{0pt}
    \setlength{\parskip}{6pt plus 2pt minus 1pt}}
}{% if KOMA class
  \KOMAoptions{parskip=half}}
\makeatother
\usepackage{xcolor}
\setlength{\emergencystretch}{3em} % prevent overfull lines
\setcounter{secnumdepth}{-\maxdimen} % remove section numbering
% Make \paragraph and \subparagraph free-standing
\ifx\paragraph\undefined\else
  \let\oldparagraph\paragraph
  \renewcommand{\paragraph}[1]{\oldparagraph{#1}\mbox{}}
\fi
\ifx\subparagraph\undefined\else
  \let\oldsubparagraph\subparagraph
  \renewcommand{\subparagraph}[1]{\oldsubparagraph{#1}\mbox{}}
\fi


\providecommand{\tightlist}{%
  \setlength{\itemsep}{0pt}\setlength{\parskip}{0pt}}\usepackage{longtable,booktabs,array}
\usepackage{calc} % for calculating minipage widths
% Correct order of tables after \paragraph or \subparagraph
\usepackage{etoolbox}
\makeatletter
\patchcmd\longtable{\par}{\if@noskipsec\mbox{}\fi\par}{}{}
\makeatother
% Allow footnotes in longtable head/foot
\IfFileExists{footnotehyper.sty}{\usepackage{footnotehyper}}{\usepackage{footnote}}
\makesavenoteenv{longtable}
\usepackage{graphicx}
\makeatletter
\def\maxwidth{\ifdim\Gin@nat@width>\linewidth\linewidth\else\Gin@nat@width\fi}
\def\maxheight{\ifdim\Gin@nat@height>\textheight\textheight\else\Gin@nat@height\fi}
\makeatother
% Scale images if necessary, so that they will not overflow the page
% margins by default, and it is still possible to overwrite the defaults
% using explicit options in \includegraphics[width, height, ...]{}
\setkeys{Gin}{width=\maxwidth,height=\maxheight,keepaspectratio}
% Set default figure placement to htbp
\makeatletter
\def\fps@figure{htbp}
\makeatother
\newlength{\cslhangindent}
\setlength{\cslhangindent}{1.5em}
\newlength{\csllabelwidth}
\setlength{\csllabelwidth}{3em}
\newlength{\cslentryspacingunit} % times entry-spacing
\setlength{\cslentryspacingunit}{\parskip}
\newenvironment{CSLReferences}[2] % #1 hanging-ident, #2 entry spacing
 {% don't indent paragraphs
  \setlength{\parindent}{0pt}
  % turn on hanging indent if param 1 is 1
  \ifodd #1
  \let\oldpar\par
  \def\par{\hangindent=\cslhangindent\oldpar}
  \fi
  % set entry spacing
  \setlength{\parskip}{#2\cslentryspacingunit}
 }%
 {}
\usepackage{calc}
\newcommand{\CSLBlock}[1]{#1\hfill\break}
\newcommand{\CSLLeftMargin}[1]{\parbox[t]{\csllabelwidth}{#1}}
\newcommand{\CSLRightInline}[1]{\parbox[t]{\linewidth - \csllabelwidth}{#1}\break}
\newcommand{\CSLIndent}[1]{\hspace{\cslhangindent}#1}

\KOMAoption{captions}{tableheading}
\makeatletter
\makeatother
\makeatletter
\makeatother
\makeatletter
\@ifpackageloaded{caption}{}{\usepackage{caption}}
\AtBeginDocument{%
\ifdefined\contentsname
  \renewcommand*\contentsname{Table of contents}
\else
  \newcommand\contentsname{Table of contents}
\fi
\ifdefined\listfigurename
  \renewcommand*\listfigurename{List of Figures}
\else
  \newcommand\listfigurename{List of Figures}
\fi
\ifdefined\listtablename
  \renewcommand*\listtablename{List of Tables}
\else
  \newcommand\listtablename{List of Tables}
\fi
\ifdefined\figurename
  \renewcommand*\figurename{Figure}
\else
  \newcommand\figurename{Figure}
\fi
\ifdefined\tablename
  \renewcommand*\tablename{Table}
\else
  \newcommand\tablename{Table}
\fi
}
\@ifpackageloaded{float}{}{\usepackage{float}}
\floatstyle{ruled}
\@ifundefined{c@chapter}{\newfloat{codelisting}{h}{lop}}{\newfloat{codelisting}{h}{lop}[chapter]}
\floatname{codelisting}{Listing}
\newcommand*\listoflistings{\listof{codelisting}{List of Listings}}
\makeatother
\makeatletter
\@ifpackageloaded{caption}{}{\usepackage{caption}}
\@ifpackageloaded{subcaption}{}{\usepackage{subcaption}}
\makeatother
\makeatletter
\@ifpackageloaded{tcolorbox}{}{\usepackage[many]{tcolorbox}}
\makeatother
\makeatletter
\@ifundefined{shadecolor}{\definecolor{shadecolor}{rgb}{.97, .97, .97}}
\makeatother
\makeatletter
\makeatother
\ifLuaTeX
  \usepackage{selnolig}  % disable illegal ligatures
\fi
\IfFileExists{bookmark.sty}{\usepackage{bookmark}}{\usepackage{hyperref}}
\IfFileExists{xurl.sty}{\usepackage{xurl}}{} % add URL line breaks if available
\urlstyle{same} % disable monospaced font for URLs
\hypersetup{
  pdftitle={Rhizaria in the oligotrophic exhibit clear temporal and vertical variability.},
  pdfauthor={Alex Barth*; Leo Blanco-Berical; Rod Johnson; Joshua Stone},
  colorlinks=true,
  linkcolor={blue},
  filecolor={Maroon},
  citecolor={Blue},
  urlcolor={Blue},
  pdfcreator={LaTeX via pandoc}}

\title{Rhizaria in the oligotrophic exhibit clear temporal and vertical
variability.}
\author{Alex Barth* \and Leo Blanco-Berical \and Rod Johnson \and Joshua
Stone}
\date{}

\begin{document}
\maketitle
\ifdefined\Shaded\renewenvironment{Shaded}{\begin{tcolorbox}[frame hidden, interior hidden, boxrule=0pt, enhanced, borderline west={3pt}{0pt}{shadecolor}, sharp corners, breakable]}{\end{tcolorbox}}\fi

\setstretch{2}
\hypertarget{introduction}{%
\section{Introduction}\label{introduction}}

Rhizaria are an extremely diverse super-group of single celled organisms
consisting of several phyla including retaria (foraminifera and
radiolaria) and cercozoa. These organisms exist in a wide range of
habitats and are widely represented in plankton communities throughout
the global ocean. While the taxonomy of these organisms has recently
undergone several reclassifications (Wiley, 2022), their presence in
ocean ecosystems has been long known to oceanographers. Some of the
earliest records of their existence are from oceanographic expeditions
in the 19th century (Haekel, 1887). Rhizaria are unique members of the
plankton and protist community because they can reach large sizes (up to
several mm in diameter) and they construct intricate mineral skeletons
out of either silica, strontium, or calcium carbonate (Kimoto, 2015;
Nakamura and Suzuki, 2015; Suzuki and Not, 2015; Wiley, 2022). Despite
their noticeable morphology and global distribution, Rhizaria were
largely understudied throughout the 20th century. The bulk of modern
plankton research has focused hard-bodied crustacea which are
numerically dominant and easily sampled with nets and preservatives.
Fragile organisms like Rhizaria, were difficult to adequately study as
they can be destroyed through standard zooplankton sampling techniques.
A number of studies in the late 1900s did employ alternative techniques
to quantify Rhizaria including diaphragm pumps (Michaels, 1988) or
blue-water SCUBA collections (Bijma et al., 1990; Caron et al., 1995;
Caron and Be, 1984). However, the bulk of Rhizaria research was
constrained to sediment traps or paleontological studies of sediment
(Boltovskoy et al., 1993; Takahashi et al., 1983). Only recently has the
advent of molecular techniques and in-situ imaging tools ignited a
renewed focus on Rhizaria in pelagic ecosystems (Caron, 2016).

The wave of new data on Rhizaria has facilitated an improved
understanding of the significance in ocean ecosystem functions. Firstly,
taxonomists have been able to greatly refine the understanding of
evolutionary relationships amongst these diverse protists (Aurahs et
al., 2009; Biard et al., 2015; Cavalier-Smith et al., 2018; Decelle et
al., 2013, 2012; rev by Wiley, 2022). DNA metabarcoding studies have
revealed insights into the distributional patterns Llopis Monferrer et
al. (2022), ecological relationships (Decelle et al., 2012; Nakamura et
al., 2023), and contribution to biogeochemical fluxes (Guidi et al.,
2016; Gutierrez-Rodriguez et al., 2019). Transcriptomic and proteomic
approaches also have been used to quantify rhizarian contribution to
community metabolism (Cohen et al., 2023). Yet, despite the excellent
taxonomic resolution provided by molecular approaches, they do not
provide a truly quantitative metric for estimating Rhizarian abundance
or biomass. In-situ imaging tools however, offer the ability to observe
organisms in the natural state and quantify their abundance (Barth and
Stone, in review; Ohman, 2019). While there were early applicaitons of
imaging tools to document Rhizaria (Dennett et al., 2002), Biard et al.
(2016)'s report from a global imaging dataset highlighted the importance
of Rhizaria to the total standing stock of marine carbon. Due to their
large sizes, ability to concentrate smaller particles (ballasting), and
the unique structure of their mineral skeletons, Rhizaria have the
potential to massively influence ocean biogeochemical cycling. A number
of studies have made large advances in estimating the contribution of
Rhizaria to ocean cycling of carbon (Gutierrez-Rodriguez et al., 2019;
Ikenoue et al., 2019; Lampitt et al., 2009; Stukel et al., 2018), silica
(Biard et al., 2018; Llopis Monferrer et al., 2021), and strontium
(Decelle et al., 2013). Still, Rhizarian ecological roles are not well
understood (Wiley, 2022). This is a major challenge as it is critical to
understand the ecological role of plankton to fully incorporate them
into biological oceanographic models.

The ecological role of Rhizaria in plankton communities is complicated
due to the fact different taxa can exhibit every different trophic
modes. As zooplankton, rhizaria are predominately heterotrophic (Wiley,
2022), yet their feeding modes can be quite varied. Phaeodarians (family
Cercozoa) are largely thought to be flux-feeders, collecting and feeding
on sinking particles (Nakamura and Suzuki, 2015; Stukel et al., 2019).
Alternatively, Retaria can be either exclusively heterotrophic or
mixotrophic, utilizing photosynthetic algal symbionts (Anderson, 2014;
Decelle et al., 2015). Mixotrophic foraminifera host a variety of
endosymbiont partners (Decelle et al., 2015; Lee, 2006), which are
thought to support early and adult life stages and contribute to total
primary productivity (Kimoto, 2015). Still, foraminifera are omnivorous,
possibly even predominately carnivorous with several studies suggesting
that they can be effective predators (Anderson and Bé, 1976; Gaskell et
al., 2019), majoritively consuming live copepods (Caron and Be, 1984).
Radiolaria have several lineages all which have some taxa who are well
known to host symbionts (Biard, 2022). Amongst radiolarians, arguably
the most widespread are Collodaria who can be either large solitary
cells or form massive colonies, up to several meters in length (Swanberg
and Anderson, 1981). All known Collodaria species host dinoflagellate
symbionts (Biard, 2022) and can contribute substantially to primary
productivity, particularly in oligotrophic ocean regions (Caron et al.,
1995; Dennett et al., 2002). This Collodaria-symbiont association has
been suggested as a reason for their high abundances throughout the
photic zone of oligotrophic environments globally (Biard et al., 2017,
2016). A few Acantharean (Radiolarian order) clades host algal symbionts
(Biard, 2022; Decelle et al., 2012), notably with two clades forming an
exclusive relationship with Phaeocystis. However, globally Acantharea
are less abundant than Collodaria (Wiley, 2022) and contribute less to
total primary productivity (Michaels et al., 1995). This may be due to
the fact several clades of Acantharea are cyst-forming and strictly
heterotrophs (Biard, 2022; Decelle et al., 2013). Furthermore, Mars
Brisbin et al. (2020) documented apparent predation behavior in
Acantharea near the surface, suggesting that there may be a large
reliance on carnivory.

Given the high abundances, yet diverse trophic strategies found among
Rhizarian taxa, it is reasonable to expect some form of niche
partitioning. A number of studies do suggest evidence for vertical
zonation between Rhizaria groups according to various trophic
strategies. Taxa-specific studies of radiolarians suggest they may be
restricted to the euphotic zone (Boltovskoy, 2017; Michaels, 1988).
Although some studies report Acantharea in deeper waters (Decelle et
al., 2013; Gutiérrez-Rodríguez et al., 2022). Phaeodarians
alternatively, are generally found in the mesopelagic where
photosynthesis cannot occur but they can feed on sinking particles
(Stukel et al., 2018). In an imaging-based study of the whole Rhizaria
community, Biard and Ohman (2020) noted clear patterns in vertical
zonation which largely corresponded to different trophic roles. In the
oligotrophic ocean, Blanco-Bercial et al. (2022) also noted that the
protist community, including Rhizaria partition along an autotroph and
mixotroph to heterotroph gradient with increasing depth in the water
column. Yet, few studies have made direct attempts to relate rhizaria
abundances to abiotic environmental factors (Biard and Ohman, 2020). In
part, this is due to the fact few studies have been able to sample
Rhizaria in the same location over a consistent timeframe (Boltovskoy et
al., 1993; Gutiérrez-Rodríguez et al., 2022; Hull et al., 2011; Michaels
et al., 1995; Michaels, 1988). Furthermore, no studies have utilized
imaging, arguably the best method for quantifying rhizaria, consistently
throughout the full mesopelagic. Given this lack of information, there
are many unknowns with respect to Rhizarian ecology, seasonality and
phenology across different groups.

In this study, we present a comprehensive assessment of large Rhizaria
measured for multiple months (greater than 1 year) using an in-situ
imaging approach. With this dataset, we address two critical aims. 1)
Quantification of large Rhizaria throughout the epipelagic (0-200m) and
mesopelagic (200-1000m) over the course of an annual cycle. These data
were collected in the Sargasso Sea, and represents the first study of
its kind in an oligotrophic system. 2) We aim to test the hypothesis
that Rhizaria exhibit niche partitioning according to trophic roles.
This hypothesis makes several predictions, including vertical zonation,
as seen in prior studies, but also that abiotic variables related to
trophic strategy will explain abundance patterns. Specifically,
autotrophic/mixotrophic taxa will correspond to variables related to
autotrophy (chl-a concentration, primary productivity, high \(O_2\) )
and other rhizaria will correspond to factors which promote heterotrophy
(particle concentration, flux, and low \(O_2\)).

\hypertarget{refs}{}
\begin{CSLReferences}{1}{0}
\leavevmode\vadjust pre{\hypertarget{ref-anderson2014}{}}%
Anderson, O.R., 2014. Living Together in the Plankton: A Survey of
Marine Protist Symbioses. Acta Protozoologica 53, 29--38.
\url{https://doi.org/10.4467/16890027AP.13.0019.1116}

\leavevmode\vadjust pre{\hypertarget{ref-anderson1976}{}}%
Anderson, O.R., Bé, A.W.H., 1976. A CYTOCHEMICAL FINE STRUCTURE STUDY OF
PHAGOTROPHY IN A PLANKTONIC FORAMINIFER, {\emph{HASTIGERINA PELAGICA}}
(d'ORBIGNY). The Biological Bulletin 151, 437--449.
\url{https://doi.org/10.2307/1540498}

\leavevmode\vadjust pre{\hypertarget{ref-aurahs2009}{}}%
Aurahs, R., Grimm, G.W., Hemleben, V., Hemleben, C., Kucera, M., 2009.
Geographical distribution of cryptic genetic types in the planktonic
foraminifer Globigerinoides ruber. Molecular Ecology 18, 1692--1706.
\url{https://doi.org/10.1111/j.1365-294X.2009.04136.x}

\leavevmode\vadjust pre{\hypertarget{ref-barth-inrev}{}}%
Barth, A., Stone, J., in review. Understanding the picture: The promis
and challenges of in-situ imagery data in the study of plankton ecology.

\leavevmode\vadjust pre{\hypertarget{ref-biard2022b}{}}%
Biard, T., 2022. Diversity and ecology of Radiolaria in modern oceans.
Environmental Microbiology 24, 2179--2200.
\url{https://doi.org/10.1111/1462-2920.16004}

\leavevmode\vadjust pre{\hypertarget{ref-biard2017}{}}%
Biard, T., Bigeard, E., Audic, S., Poulain, J., Gutierrez-Rodriguez, A.,
Pesant, S., Stemmann, L., Not, F., 2017. Biogeography and diversity of
Collodaria (Radiolaria) in the global ocean. The ISME Journal 11,
1331--1344. \url{https://doi.org/10.1038/ismej.2017.12}

\leavevmode\vadjust pre{\hypertarget{ref-biard2018}{}}%
Biard, T., Krause, J.W., Stukel, M.R., Ohman, M.D., 2018. The
Significance of Giant Phaeodarians (Rhizaria) to Biogenic Silica Export
in the California Current Ecosystem. Global Biogeochemical Cycles 32,
987--1004. \url{https://doi.org/10.1029/2018GB005877}

\leavevmode\vadjust pre{\hypertarget{ref-biard2020}{}}%
Biard, T., Ohman, M.D., 2020. Vertical niche definition of test-bearing
protists (Rhizaria) into the twilight zone revealed by in situ imaging.
Limnology and Oceanography 65, 2583--2602.
\url{https://doi.org/10.1002/lno.11472}

\leavevmode\vadjust pre{\hypertarget{ref-biard2015}{}}%
Biard, T., Pillet, L., Decelle, J., Poirier, C., Suzuki, N., Not, F.,
2015. Towards an integrative morpho-molecular classification of the
collodaria (polycystinea, radiolaria). Protist 166, 374--388.
\url{https://doi.org/10.1016/j.protis.2015.05.002}

\leavevmode\vadjust pre{\hypertarget{ref-biard2016}{}}%
Biard, T., Stemmann, L., Picheral, M., Mayot, N., Vandromme, P., Hauss,
H., Gorsky, G., Guidi, L., Kiko, R., Not, F., 2016. In situ imaging
reveals the biomass of giant protists in the global ocean. Nature 532,
504--507. \url{https://doi.org/10.1038/nature17652}

\leavevmode\vadjust pre{\hypertarget{ref-bijma1990}{}}%
Bijma, J., Erez, J., Hemleben, C., 1990.
\href{https://epic.awi.de/id/eprint/6096/1/Bij1990a.pdf}{Lunar and
semi-luan reproductive cycles in some spinose planktonic foraminifers}.
Journal of Foraminiferal Research 20, 117--127.

\leavevmode\vadjust pre{\hypertarget{ref-blanco-bercial2022}{}}%
Blanco-Bercial, L., Parsons, R., Bolaños, L.M., Johnson, R., Giovannoni,
S.J., Curry, R., 2022.
\href{https://www.frontiersin.org/articles/10.3389/fmars.2022.897140}{The
protist community traces seasonality and mesoscale hydrographic features
in the oligotrophic sargasso sea}. Frontiers in Marine Science 9.

\leavevmode\vadjust pre{\hypertarget{ref-boltovskoy2017}{}}%
Boltovskoy, D., 2017. Vertical distribution patterns of radiolaria
polycystina (protista) in the world ocean: Living ranges, isothermal
submersion and settling shells. Journal of Plankton Research 39,
330--349. \url{https://doi.org/10.1093/plankt/fbx003}

\leavevmode\vadjust pre{\hypertarget{ref-boltovskoy1993}{}}%
Boltovskoy, D., Alder, V.A., Abelmann, A., 1993. Annual flux of
radiolaria and other shelled plankters in the eastern equatorial
atlantic at 853 m: seasonal variations and polycystine species-specific
responses. Deep Sea Research Part I: Oceanographic Research Papers 40,
1863--1895. \url{https://doi.org/10.1016/0967-0637(93)90036-3}

\leavevmode\vadjust pre{\hypertarget{ref-caron2016}{}}%
Caron, D.A., 2016.
\href{https://www.nature.com/articles/nature17892}{The rise of rhizaria
\textbar{} nature}. Nature 532, 444--445.

\leavevmode\vadjust pre{\hypertarget{ref-caron1984}{}}%
Caron, D.A., Be, A.W.H., 1984. PREDICTED AND OBSERVED FEEDING RATES OF
THE SPINOSE PLANKTONIC FORAMINIFER. BULLETIN OF MARINE SCIENCE 35.

\leavevmode\vadjust pre{\hypertarget{ref-caron1995}{}}%
Caron, D.A., Michaels, A.F., Swanberg, N.R., Howse, F.A., 1995. Primary
productivity by symbiont-bearing planktonic sarcodines (acantharia,
radiolaria, foraminifera) in surface waters near bermuda. Journal of
Plankton Research 17, 103--129.
\url{https://doi.org/10.1093/plankt/17.1.103}

\leavevmode\vadjust pre{\hypertarget{ref-cavalier-smith2018}{}}%
Cavalier-Smith, T., Chao, E.E., Lewis, R., 2018. Multigene phylogeny and
cell evolution of chromist infrakingdom Rhizaria: contrasting cell
organisation~of sister~phyla Cercozoa and Retaria. Protoplasma 255,
1517--1574. \url{https://doi.org/10.1007/s00709-018-1241-1}

\leavevmode\vadjust pre{\hypertarget{ref-cohen2023}{}}%
Cohen, N.R., Krinos, A.I., Kell, R.M., Chmiel, R.J., Moran, D.M.,
McIlvin, M.R., Lopez, P.Z., Barth, A., Stone, J., Alanis, B.A., Chan,
E.W., Breier, J.A., Jakuba, M.V., Johnson, R., Alexander, H., Saito,
M.A., 2023. Microeukaryote metabolism across the western North Atlantic
Ocean revealed through autonomous underwater profiling.
\url{https://doi.org/10.1101/2023.11.20.567900}

\leavevmode\vadjust pre{\hypertarget{ref-decelle2015}{}}%
Decelle, J., Colin, S., Foster, R.A., 2015. Photosymbiosis in Marine
Planktonic Protists, in: Ohtsuka, S., Suzaki, T., Horiguchi, T., Suzuki,
N., Not, F. (Eds.),. Springer Japan, Tokyo, pp. 465--500.
\url{https://doi.org/10.1007/978-4-431-55130-0_19}

\leavevmode\vadjust pre{\hypertarget{ref-decelle2013}{}}%
Decelle, J., Martin, P., Paborstava, K., Pond, D.W., Tarling, G., Mahé,
F., De Vargas, C., Lampitt, R., Not, F., 2013. Diversity, Ecology and
Biogeochemistry of Cyst-Forming Acantharia (Radiolaria) in the Oceans.
PLoS ONE 8, e53598. \url{https://doi.org/10.1371/journal.pone.0053598}

\leavevmode\vadjust pre{\hypertarget{ref-decelle2012}{}}%
Decelle, J., Siano, R., Probert, I., Poirier, C., Not, F., 2012.
Multiple microalgal partners in symbiosis with the acantharian
Acanthochiasma sp. (Radiolaria). Symbiosis 58, 233--244.
\url{https://doi.org/10.1007/s13199-012-0195-x}

\leavevmode\vadjust pre{\hypertarget{ref-dennett2002}{}}%
Dennett, M.R., Caron, D.A., Michaels, A.F., Gallager, S.M., Davis, C.S.,
2002. Video plankton recorder reveals high abundances of colonial
radiolaria in surface waters of the central north pacific. Journal of
Plankton Research 24, 797--805.
\url{https://doi.org/10.1093/plankt/24.8.797}

\leavevmode\vadjust pre{\hypertarget{ref-gaskell2019}{}}%
Gaskell, D.E., Ohman, M.D., Hull, P.M., 2019. Zooglider-based
measurements of planktonic foraminifera in the california current
system. Journal of Foraminiferal Research 49, 390--404.
\url{https://doi.org/10.2113/gsjfr.49.4.390}

\leavevmode\vadjust pre{\hypertarget{ref-guidi2016}{}}%
Guidi, L., Chaffron, S., Bittner, L., Eveillard, D., Larhlimi, A., Roux,
S., Darzi, Y., Audic, S., Berline, L., Brum, J.R., Coelho, L.P.,
Espinoza, J.C.I., Malviya, S., Sunagawa, S., Dimier, C., Kandels-Lewis,
S., Picheral, M., Poulain, J., Searson, S., Stemmann, L., Not, F.,
Hingamp, P., Speich, S., Follows, M., Karp-Boss, L., Boss, E., Ogata,
H., Pesant, S., Weissenbach, J., Wincker, P., Acinas, S.G., Bork, P.,
Vargas, C. de, Iudicone, D., Sullivan, M.B., Raes, J., Karsenti, E.,
Bowler, C., Gorsky, G., 2016. Plankton networks driving carbon export in
the oligotrophic ocean. Nature 532, 465--470.
\url{https://doi.org/10.1038/nature16942}

\leavevmode\vadjust pre{\hypertarget{ref-gutierrez-rodriguez2019}{}}%
Gutierrez-Rodriguez, A., Stukel, M.R., Lopes dos Santos, A., Biard, T.,
Scharek, R., Vaulot, D., Landry, M.R., Not, F., 2019. High contribution
of Rhizaria (Radiolaria) to vertical export in the California Current
Ecosystem revealed by DNA metabarcoding. The ISME Journal 13, 964--976.
\url{https://doi.org/10.1038/s41396-018-0322-7}

\leavevmode\vadjust pre{\hypertarget{ref-gutiuxe9rrez-rodruxedguez2022}{}}%
Gutiérrez-Rodríguez, A., Lopes dos Santos, A., Safi, K., Probert, I.,
Not, F., Fernández, D., Gourvil, P., Bilewitch, J., Hulston, D.,
Pinkerton, M., Nodder, S.D., 2022. Planktonic protist diversity across
contrasting subtropical and subantarctic waters of the southwest
pacific. Progress in Oceanography 206, 102809.
\url{https://doi.org/10.1016/j.pocean.2022.102809}

\leavevmode\vadjust pre{\hypertarget{ref-haekel1887}{}}%
Haekel, E., 1887.
\href{https://www.gutenberg.org/cache/epub/44527/pg44527-images.html}{Report
on the radiolaria collected by h.m.s. Challenger during the years
1873-1876, plates, by ernst haeckel}.

\leavevmode\vadjust pre{\hypertarget{ref-hull2011}{}}%
Hull, P.M., Osborn, K.J., Norris, R.D., Robison, B.H., 2011. Seasonality
and depth distribution of a mesopelagic foraminifer, Hastigerinella
digitata, in Monterey Bay, California. Limnology and Oceanography 56,
562--576. \url{https://doi.org/10.4319/lo.2011.56.2.0562}

\leavevmode\vadjust pre{\hypertarget{ref-ikenoue2019}{}}%
Ikenoue, T., Kimoto, K., Okazaki, Y., Sato, M., Honda, M.C., Takahashi,
K., Harada, N., Fujiki, T., 2019. Phaeodaria: An Important Carrier of
Particulate Organic Carbon in the Mesopelagic Twilight Zone of the North
Pacific Ocean. Global Biogeochemical Cycles 33, 1146--1160.
\url{https://doi.org/10.1029/2019GB006258}

\leavevmode\vadjust pre{\hypertarget{ref-kimoto2015}{}}%
Kimoto, K., 2015. Planktic foraminifera. pp. 129--178.
\url{https://doi.org/10.1007/978-4-431-55130-0_7}

\leavevmode\vadjust pre{\hypertarget{ref-lampitt2009}{}}%
Lampitt, R.S., Salter, I., Johns, D., 2009. Radiolaria: Major exporters
of organic carbon to the deep ocean. Global Biogeochemical Cycles 23.
\url{https://doi.org/10.1029/2008GB003221}

\leavevmode\vadjust pre{\hypertarget{ref-lee2006}{}}%
Lee, J.J., 2006.
\href{https://dalspace.library.dal.ca/bitstream/handle/10222/78255/VOLUME\%2042-NUMBER\%202-2006-PAGE\%2063.pdf?sequence=1}{Algal
symbiosis in larger foraminifera}. Symbiosis 42, 63--75.

\leavevmode\vadjust pre{\hypertarget{ref-llopismonferrer2022}{}}%
Llopis Monferrer, N., Biard, T., Sandin, M.M., Lombard, F., Picheral,
M., Elineau, A., Guidi, L., Leynaert, A., Tréguer, P.J., Not, F., 2022.
\href{https://www.frontiersin.org/articles/10.3389/fmars.2022.895995}{Siliceous
rhizaria abundances and diversity in the mediterranean sea assessed by
combined imaging and metabarcoding approaches}. Frontiers in Marine
Science 9.

\leavevmode\vadjust pre{\hypertarget{ref-llopismonferrer2021}{}}%
Llopis Monferrer, N., Leynaert, A., Tréguer, P., Gutiérrez-Rodríguez,
A., Moriceau, B., Gallinari, M., Latasa, M., L'Helguen, S., Maguer,
J.-F., Safi, K., Pinkerton, M.H., Not, F., 2021. Role of small Rhizaria
and diatoms in the pelagic silica production of the Southern Ocean.
Limnology and Oceanography 66, 2187--2202.
\url{https://doi.org/10.1002/lno.11743}

\leavevmode\vadjust pre{\hypertarget{ref-marsbrisbin2020}{}}%
Mars Brisbin, M., Brunner, O.D., Grossmann, M.M., Mitarai, S., 2020.
Paired high-throughput, in situ imaging and high-throughput sequencing
illuminate acantharian abundance and vertical distribution. Limnology
and Oceanography 65, 2953--2965. \url{https://doi.org/10.1002/lno.11567}

\leavevmode\vadjust pre{\hypertarget{ref-michaels1988}{}}%
Michaels, A.F., 1988. Vertical distribution and abundance of Acantharia
and their symbionts. Marine Biology 97, 559--569.
\url{https://doi.org/10.1007/BF00391052}

\leavevmode\vadjust pre{\hypertarget{ref-michaels1995}{}}%
Michaels, A.F., Caron, D.A., Swanberg, N.R., Howse, F.A., Michaels,
C.M., 1995. Planktonic sarcodines (Acantharia, Radiolaria, Foraminifera)
in surface waters near Bermuda: abundance, biomass and vertical flux.
Journal of Plankton Research 17, 131--163.
\url{https://doi.org/10.1093/plankt/17.1.131}

\leavevmode\vadjust pre{\hypertarget{ref-nakamura2023}{}}%
Nakamura, Y., Itagaki, H., Tuji, A., Shimode, S., Yamaguchi, A., Hidaka,
K., Ogiso-Tanaka, E., 2023. DNA metabarcoding focused on
difficult-to-culture protists: An effective approach to clarify
biological interactions. Environmental Microbiology 25, 3630--3638.
\url{https://doi.org/10.1111/1462-2920.16524}

\leavevmode\vadjust pre{\hypertarget{ref-nakamura2015}{}}%
Nakamura, Y., Suzuki, N., 2015.
\href{https://link.springer.com/chapter/10.1007/978-4-431-55130-0_9}{Phaeodaria:
Diverse marine cercozoans of world-wide distribution}. Springer, pp.
223--249.

\leavevmode\vadjust pre{\hypertarget{ref-ohman2019a}{}}%
Ohman, M.D., 2019. A sea of tentacles: optically discernible traits
resolved from planktonic organisms in situ. ICES Journal of Marine
Science 76, 1959--1972. \url{https://doi.org/10.1093/icesjms/fsz184}

\leavevmode\vadjust pre{\hypertarget{ref-sogawa2022}{}}%
Sogawa, S., Nakamura, Y., Nagai, S., Nishi, N., Hidaka, K., Shimizu, Y.,
Setou, T., 2022. DNA metabarcoding reveals vertical variation and hidden
diversity of alveolata and rhizaria communities in the western north
pacific. Deep Sea Research Part I: Oceanographic Research Papers 183,
103765. \url{https://doi.org/10.1016/j.dsr.2022.103765}

\leavevmode\vadjust pre{\hypertarget{ref-stukel2018}{}}%
Stukel, M.R., Biard, T., Krause, J., Ohman, M.D., 2018. Large Phaeodaria
in the twilight zone: Their role in the carbon cycle. Limnology and
Oceanography 63, 2579--2594. \url{https://doi.org/10.1002/lno.10961}

\leavevmode\vadjust pre{\hypertarget{ref-stukel2019}{}}%
Stukel, M.R., Ohman, M.D., Kelly, T.B., Biard, T., 2019.
\href{https://www.frontiersin.org/articles/10.3389/fmars.2019.00397}{The
roles of suspension-feeding and flux-feeding zooplankton as gatekeepers
of particle flux into the mesopelagic ocean in the northeast pacific}.
Frontiers in Marine Science 6.

\leavevmode\vadjust pre{\hypertarget{ref-suzuki2015}{}}%
Suzuki, N., Not, F., 2015.
\href{https://link.springer.com/chapter/10.1007/978-4-431-55130-0_8}{Biology
and ecology of radiolaria}. Springer, pp. 179--222.

\leavevmode\vadjust pre{\hypertarget{ref-swanberg1981}{}}%
Swanberg, N.R., Anderson, O.R., 1981. Collozoum caudatum sp. nov.: A
giant colonial radiolarian from equatorial and Gulf Stream waters. Deep
Sea Research Part A. Oceanographic Research Papers 28, 1033--1047.
\url{https://doi.org/10.1016/0198-0149(81)90016-9}

\leavevmode\vadjust pre{\hypertarget{ref-takahashi1983}{}}%
Takahashi, K., Hurd, D.C., Honjo, S., 1983. Phaeodarian skeletons: Their
role in silica transport to the deep sea. Science 222, 616--618.
\url{https://doi.org/10.1126/science.222.4624.616}

\leavevmode\vadjust pre{\hypertarget{ref-biard2022}{}}%
Wiley, 2022. Rhizaria. Wiley, pp. 1--11.
\url{https://doi.org/10.1002/9780470015902.a0029469}

\end{CSLReferences}



\end{document}
